%%%%%%%%%%%%%%%%%%%%%%%%%%%%%%%%%%%%%%%%%
% University/School Laboratory Report
% LaTeX Template
% Version 3.1 (25/3/14)
%
% This template has been downloaded from:
% http://www.LaTeXTemplates.com
%
% Original author:
% Linux and Unix Users Group at Virginia Tech Wiki 
% (https://vtluug.org/wiki/Example_LaTeX_chem_lab_report)
%
% License:
% CC BY-NC-SA 3.0 (http://creativecommons.org/licenses/by-nc-sa/3.0/)
%
%%%%%%%%%%%%%%%%%%%%%%%%%%%%%%%%%%%%%%%%%

%----------------------------------------------------------------------------------------
%	PACKAGES AND DOCUMENT CONFIGURATIONS
%----------------------------------------------------------------------------------------

\documentclass{article}

\usepackage[version=3]{mhchem} % Package for chemical equation typesetting
\usepackage{siunitx} % Provides the \SI{}{} and \si{} command for typesetting SI units
\usepackage{graphicx} % Required for the inclusion of images
\usepackage{natbib} % Required to change bibliography style to APA
\usepackage{amsmath} % Required for some math elements 

\usepackage[utf8]{inputenc}
\usepackage[T1]{fontenc}

\setlength\parindent{0pt} % Removes all indentation from paragraphs
\setlength{\parskip}{1em}
\renewcommand{\labelenumi}{\alph{enumi}.} % Make numbering in the enumerate environment by letter rather than number (e.g. section 6)

%\usepackage{times} % Uncomment to use the Times New Roman font

%----------------------------------------------------------------------------------------
%	DOCUMENT INFORMATION
%----------------------------------------------------------------------------------------

\title{Laboration 1 \\ SystemC TrafficLight} % Title

\author{
  Björn \textsc{Hvass}\\
  Cyril \textsc{Barrelet}
} % Authors name

\date{\today} % Date for the report

\begin{document}

\maketitle % Insert the title, author and date

\begin{center}
\begin{tabular}{l r}
Course: & TDTS07\\ % Date the experiment was performed
Liu Ids: & Hvass bjohv276\\ % Partner names
& Barrelet cyrba593 \\
\end{tabular}
\end{center}

% If you wish to include an abstract, uncomment the lines below
% \begin{abstract}
% Abstract text
% \end{abstract}

\newpage\section{Sensor module}
The purpose of the sensor module is to keep track of how many cars are waiting for a traffic light. To do this it has one increment method and one decrement thread. Furthermore, it has two input signals. The first is from the generator that is used by the increment method. The second is from the controller and is used by the decrement thread. The module also has an output signal to the controller module.

\subsection{Sensor increment method}

Every cars sent by the generator will trigg the sensor method and the method will basically increment the number of cars in the concerned direction.   

\subsection{Sensor decrement thread}

At first, wich two seconds, we rewrite the output of the module with 'false' if there is no cars waitting in the queue or with 'true' if there are. 

And finally, we decrement the number of cars if the concerned traffic light is green. 

\newpage\section{Generator module}

The generator module simulates cars in order to show the different properties of the lights. 
To prove those properties, two differents modules has been created.
The first one generates pseudo-random cars to show the behaviour of the lights during a time period.
The other one generates cars by reading a text file file to show all possible cases, one by one. 

The generator contains a print method, a generator method and a thread to trigger the generator method.

\subsection{Event thread}

The event{\_}creator is a thread that triggers the generator method periodically. The generator output will be '1' if there is a car or  '0' if there isn't.

\subsection{Random generator method}

For every events, the module has a 30{\%} chance of generating a car.

The toggle is used to invoke a change in the channel when there is a car. The change is used by the sensor to trigger its increment method.

\subsection{Generator method}

This module reads a dedicated text to show the different properties. 

All cases are presented below: 

\begin{table}[]
\begin{tabular}{llllllllllllllllll}
N & : & 1 & 0 & 1 & 0 & 1 & 0 & 1 & 0 & 1 & 0 & 1 & 0 & 1 & 0 & 1 & 0 \\
S & : & 0 & 1 & 1 & 0 & 0 & 1 & 1 & 0 & 0 & 1 & 1 & 0 & 0 & 1 & 1 & 0 \\
E & : & 0 & 0 & 0 & 1 & 1 & 1 & 1 & 0 & 0 & 0 & 0 & 1 & 1 & 1 & 1 & 0 \\
W & : & 0 & 0 & 0 & 0 & 0 & 0 & 0 & 1 & 1 & 1 & 1 & 1 & 1 & 1 & 1 & 0 \\
  &   &   &   &   &   &   &   &   &   &   &   &   &   &   &   &   &   \\
  &   & N & - & N & - & N & - & N & - & N & - & N & - & N & - & N & - \\
  &   & - & S & S & - & - & S & S & - & - & S & S & - & - & S & S & - \\
  &   & - & - & - & E & E & E & E & - & - & - & - & E & E & E & E & - \\
  &   & - & - & - & - & - & - & - & W & W & W & W & W & W & W & W & -
\end{tabular}
\end{table}

\newpage\section{Section Name}
% Write about the purpose of the Module
% Explain the module signals


\subsection{Function 1}
%Purpose of function and why we implemented it in the way we did
%Skip the print_method and write about it generally in the end.


%test for showcase purpose stuffs

\newpage%Explain how to system is connected and work together
%maybe a schematic and a general overview
%Do this in the end like a summery to tie the report together

\newpage%Explain the tests that we do and why we do them.
%Look what the lab has to be enabled to do according to the instructions and explain why our test dose that.

\section{How to run}
To run the system with the test files first make the project with < make -f MakeFile > and then <main.x fil1 file2 file3 file4 fileOut>. The files are input files for the directions with 0 representing no car and 1 represents a car. The last file is an output file. To compile the project to generate pseudo-random cars, use <make -f MakeFile_Random> and run <main.x> fs


%----------------------------------------------------------------------------------------


\end{document}
