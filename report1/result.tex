%Explain the tests that we do and why we do them.
%Look what the lab has to be enabled to do according to the instructions and explain why our test dose that.

\section{How to run}
To run the system with the test files first make the project with < make -f MakeFile > and then <main.x fil1 file2 file3 file4 fileOut>. The files are input files for the directions with 0 representing no car and 1 represents a car. The last file is an output file. To compile the project to generate pseudo-random cars, use <make -f MakeFile_Random> and run <main.x> fs

\section{Independence and Safety}

The proof that the lights work independently and safety is demonstrated by the test bench. This test bench, given the test files, represents all sixteen possible cases.

For example, if the light NS is green, the light SN is red if there are no cars coming in the direction SN.
Furthermore, the North or South lights can't be green at the same time as the East or West lights.
And even more, if a vehicle arrives at the crossing, it will eventually be granted the green light, according to the safety property.

This result is possible thanks to a state machine or ´´controller''' who prevents any problem by reading the sensors and set the lights properly. 

\section{Conclusion}

After all those properties have been proved, the random car generation can ensure that the system works correctly in the time. 
