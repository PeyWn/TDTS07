%%%%%%%%%%%%%%%%%%%%%%%%%%%%%%%%%%%%%%%%%
% University/School Laboratory Report
% LaTeX Template
% Version 3.1 (25/3/14)
%
% This template has been downloaded from:
% http://www.LaTeXTemplates.com
%
% Original author:
% Linux and Unix Users Group at Virginia Tech Wiki 
% (https://vtluug.org/wiki/Example_LaTeX_chem_lab_report)
%
% License:
% CC BY-NC-SA 3.0 (http://creativecommons.org/licenses/by-nc-sa/3.0/)
%
%%%%%%%%%%%%%%%%%%%%%%%%%%%%%%%%%%%%%%%%%

%----------------------------------------------------------------------------------------
%	PACKAGES AND DOCUMENT CONFIGURATIONS
%----------------------------------------------------------------------------------------

\documentclass{article}

\usepackage[version=3]{mhchem} % Package for chemical equation typesetting
\usepackage{siunitx} % Provides the \SI{}{} and \si{} command for typesetting SI units
\usepackage{graphicx} % Required for the inclusion of images
\usepackage{natbib} % Required to change bibliography style to APA
\usepackage{amsmath} % Required for some math elements 

\setlength\parindent{0pt} % Removes all indentation from paragraphs
\setlength{\parskip}{1em}
\renewcommand{\labelenumi}{\alph{enumi}.} % Make numbering in the enumerate environment by letter rather than number (e.g. section 6)

%\usepackage{times} % Uncomment to use the Times New Roman font

%----------------------------------------------------------------------------------------
%	DOCUMENT INFORMATION
%----------------------------------------------------------------------------------------

\title{Laboration 1 \\ SystemC TrafficLight} % Title

\author{
  Björn \textsc{Hvass}\\
  Cyril \textsc{Barrelet}
} % Authors name

\date{\today} % Date for the report

\begin{document}

\maketitle % Insert the title, author and date

\begin{center}
\begin{tabular}{l r}
Course: & TDTS07\\ % Date the experiment was performed
Liu Ids: & Hvass bjohv276\\ % Partner names
& Barrelet cyrba593 \\
\end{tabular}
\end{center}

% If you wish to include an abstract, uncomment the lines below
% \begin{abstract}
% Abstract text
% \end{abstract}
\pagebreak
\pagebreak\section{Getting started}
\label{sec:org7b4758c}

For the first query E<> P.s3 witch means that there exist a possible future where P.s3 holds. The result is that the property is satisfied witch is true since there is a possible sequence of state transitions leading form the initial state to s3.
An example sequence : S0 $\rightarrow$ S1 $\rightarrow$ S3

For the second query A<> P.s3 witch means that all possible sequence of states that eventually leads to P.s3. The property is not satisfied since there is no way to ensure that there is such a sequence. This is due to the fact that there is nothing forcing the states to change.
An example sequence: S0     The state machine will stay in S0 for infinity. 


\pagebreak\section{Fisher 1}
\label{sec:orga07fe9b}
For this assignment we try to predict the verification time for fisher with n = 12. We used a android phone to measure time since uppaal doesn't report it.

\begin{itemize}
\item n : 8     1s
\item n : 9     2s
\item n : 10    6s
\item n : 11    20s
\item n : 12    120s (Our guess)
\end{itemize}

It seems to be a factorial increment of time.


\pagebreak\section{Fisher 2}
\label{sec:org1637cd2}
If m < k then the mutex requirement will not be satisfied. An example of this is the sequence:

(-,-,-,-) $\rightarrow$ (req,-,-,-) $\rightarrow$ (req,req,-,-) $\rightarrow$ (wait,req,-,-) $\rightarrow$ (cs,req,-,-) $\rightarrow$ (cs,wait,-,-) $\rightarrow$ (cs,cs,-,-)

If m  \textgreater= k then mutex requirement will be satisfied.

 

\pagebreak\subsection{Traffic Light Controller}
\label{sec:org7954bf6}
The Traffic light timed automata to have three templates. Two for the traffic lights and one controller. The reason that we chose to have two templates for the traffic light instead of one is that there is a slight difference between north and south compared to east and west. 

We have an integer called transfer that behaves like a four+bit number. We use this to define seven states of the system, north, south, east, west, north-south, east-west or none. To do this each traffic direction is initialized with an id, 1,2,4,8; these are then added and subtracted form transfer when a light goes to and from green respectively. This enables us to define states be checking the value of transfer, if it's less the 4, for example, the state is 1\&2. Less then 2 the state is 1 and so on.

\subsection{Traffic Light Template}
\label{sec:orgc8497f6}
The template has three states, \emph{idle} (red light), \emph{wait} (there are cars waiting) and \emph{Green} (green light). To force the system to transit state the edge form \emph{wait} to \emph{Green} is triggered be a channel form the controller while \emph{Green} has a timed invariant. 

\subsection{Controller}
\label{sec:org0c24095}
The controller has to states that it switches between on a timed interval. The edges form NS to WE triggers one channel for the WE traffic light and vice versa for the edge form WE to NS but with another channel.




%Bib
\bibliographystyle{apalike}

\bibliography{sample}

%----------------------------------------------------------------------------------------


\end{document}
